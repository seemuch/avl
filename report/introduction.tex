\section{Introduction}

\subsection{The Problem}
   
AVL (Algorithm Visualization Language) is an intuitive educational programming language that
concentrates on the algorithm teaching and learning in the way of visualization. The output of AVL
is a sequence of graphics or video, which describes how an algorithm works. AVL provides varying
degrees of flexibility and programmability. It helps beginners understand a complicated algorithm in
an intuitive way. It assists professionals to visualize their self-design algorithms, where AVL
provides visualization control interface.
    
AVL breaks the limitation of current similar languages. It is highly focused on the robust algorithm
guarantee as a visualization language. The shining breakthrough is to provide a series of
complicated low-level drawing operations, which is specially designed for algorithm visualization.
	 
AVL is also an educational language. It provides simple and clear grammar with C++ style. Benefiting
from the easy-to-use features of AVL, the target users, teachers and students, are able to easily
demonstrate or learn the algorithms.
	  
	   
\subsection{Why Use AVL}
	    
There are several existing approaches to algorithm visualization. First, in order to visualize
algorithms such as Quicksort, we can just write a C++ program that manipulates an array and use
third-party libraries to display the animation. Second, there are lots of videos which show
different kinds of algorithms and data structures, such as sorting
(\url{http://www.sorting-algorithms.com/}) and trees
(\url{http://www.qmatica.com/DataStructures/Trees/AVL/AVLTree.html}). There are also some
visualization techniques, such as Animal (\url{http://www.algoanim.info/AnimalAV/}). However, these
approaches all have certain kinds of drawbacks and limitations.
		 
\subsubsection{Complicated low-level drawing operations}

Some approaches require users to implement the low-level drawing operations. People have to generate
the video frame by frame, take into account where to put the array on the screen and think about
what third-party library is available to draw the variables and operations. These low-level
operations are complicated, making algorithm visualization difficult and inefficient.
		  
\subsubsection{Limitation on available algorithms}

Many approaches only allow visualization of a limited number of existing algorithms. They do not
allow user to visualize their own algorithms.
		   
\subsubsection{Hard to learn and use}

Some approaches, such as Animal (\url{http://www.algoanim.info/AnimalAV/}), provides a good
animation tool for general-purpose animation. However, learning to use these tools is also hard and
takes extra time for people who only have a basic understanding of C programming.
		    
			 
\subsection{Target Users}
			  
The target users of our algorithm visualization language are teachers and students in computer
education in high school. This language can also be utilized in basic programming courses in college
like Data Structure.
			   
Teachers with rich programming experiences can use our AVL to generate the video for the algorithms
they prepare to demonstrate in class. With our language, they do not need to draw the picture
describing their algorithm in the old fashioned way. In addition, because our languages are highly
flexible, teachers can visualize whatever algorithm they want. For example, when preparing a
teaching material, you may easily get the picture for Quick Sort from internet. But it’s difficult
to get one for Randomized Quick Sort. Then our AVL may help you at this time. You just need to write
a Randomized Quick Sort liked program. Our language will generate the video for you automatically.
			    
Our language is also a good tool for the students or beginner in coding. Students with no
programming background can use our built-in library such as Stack and LinkedList. Operation of these
built-in data structures are well encapsulated, but we can generate the video telling you what is
happening inside.
				 
				  
\subsection{Properties}
				   
\subsubsection{Visualization}

According to research, visual information accounts for more than $70\%$ of all the information human
perceive. And that is our origin motivation of inventing AVL. The most significant feature of AVL
is, as its name indicates, visualization. Instead of focusing on the outcome of algorithms as normal
programming languages do, AVL concentrates on the process of algorithms. With AVL, users can
actually see how their programs run, via which they can have a better understanding of their
programs and algorithms.
				    
\subsubsection{Convenient}

The primary goal of AVL is to hide the complex visualization process behind the code for user to
write. By using AVL, user can focus on the algorithm and write C-style code without thinking how to
visualize it. The visualization process, such as where to draw the array on the screen and how to
highlight a part of a array, is generated automatically when the code is compiled.
					 
\subsubsection{Flexible}

The most important feature of AVL is visualization. But users of AVL may not what to visualize
everything they write. Therefore, AVL has some key words to let the uses define which part of their
program to visualize. It gives the users a flexible choice to generate their own animations. For
example, some user may be curious about the difference between merge sort and insertion sort, and he
can create two animations with AVL to compare them. While in other cases, sorting might just be a
subroutine of a complex program, and thus can be ignored in the animation.
					  
\subsubsection{Simple and Familiar}

Since the users of the AVL are professors and students who teach and learn algorithms, respectively,
we want to design AVL as a simple language that can be used without extensive training.

We keep AVL as a similar language to C++ which means that the programmers can migrate easily to our
language. Focusing on the problem of algorithm visualization, we must add new features such as
keywords mentioned in part ``Flexible''. Moreover, considering the fact that student users may be
entry level programmers and the limitation of visualization, we remove some unnecessary complexities
of C++ and add some build-in libraries which include general data structures and algorithms for the
beginners.
					   
\subsubsection{Robust}

AVL is designed for creating clearly and understandable animations of algorithms, so it provides
compile-time checking of the code to prevent violations such as exceeding array bounds. However, for
the purpose of flexibility, AVL cannot check whether the program is ``correct''. For example, the
user may write a sort algorithm whose result is still unsorted, or the user may write an infinite
loop which makes the program never end. To solve this problem, we simply create the animation as the
user defined in their program and let them to find the error by themselves, which, on the other
hand, makes AVL a debug tool for students learning algorithms.
