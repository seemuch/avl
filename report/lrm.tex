\section{Language Reference Manual}

\subsection{Introduction}

This manual introduces abundant features of AVL visualizing algorithms with OpenGL easier. We
firstly describe lexical conventions used in this language. Then we introduce the language syntax,
and finally end with a grammar to represent AVL.

\subsection{Lexical Conventions}

Each program is reduced to a stream of tokens. Each token can be one of the following six classes:
identifiers, keywords, constants, operators, separators, and whitespace. Whitespace is the
collective term used for spaces, tabs, newlines, form-feeds, and comments.

\subsubsection{Comments}

In AVL we only support multiline comments defined as word between \verb"/*" and \verb"*/", nested
comments are not allowed.

\subsubsection{Identifiers}

Identifiers are sequences of characters used for naming variables, and functions. Letters, decimal
digits, and the underscore character can be included in identifiers. An identifier cannot start with
a digit. Any letters used in identifiers are case-sensitive.

\subsubsection{Keywords}

The following identifiers are reserved as keywords, which cannot be used for any other purpose.

\begin{table}[htp]
  \centering
  \begin{tabular}{llll}
    \verb"bool" \hspace{2cm} & \verb"else" \hspace{2cm} & \verb"int" \hspace{2cm} & \verb"void" \\
    \verb"break" & \verb"false" & \verb"len" & \verb"while" \\
    \verb"char" & \verb"for" & \verb"return" & \verb"<begin_display>" \\
		\verb"continue" & \verb"hide" & \verb"string" & \verb"<end_display>" \\
		\verb"display" & \verb"if" & \verb"swap" & \verb"do" \\
		\verb"index" & \verb"true"
  \end{tabular}
  \caption{Keywords}
  \label{tab:keywords}
\end{table}

\subsubsection{Reserved}
The following characters are reserved for use in the grammar.

\begin{table}[htp]
  \centering
  \begin{tabular}{lllll}
    \verb"&&" \hspace{1cm} & \verb"!=" \hspace{1cm} & \verb"(" \hspace{1cm} & \verb"]" \hspace{1cm} & \verb"/" \\
    \verb"||" & \verb";" & \verb")" & \verb"!" & \verb"<" \\
    \verb"<=" & \verb"," & \verb"{" & \verb"-" & \verb">" \\
    \verb">=" & \verb":" & \verb"}" & \verb"+" & \verb"++" \\
    \verb"==" & \verb"=" & \verb"[" & \verb"*" & \verb"--"
  \end{tabular}
  \caption{Reserved characters}
  \label{tab:reserved}
\end{table}


\subsection{Types}

\subsubsection{Basic Types}

AVL has several primitive types.

\paragraph{int}
The 32-bit signed int data type holds integer values in the range of -2,147,483,648 to
2,147,483,647.

\paragraph{char}
A char is used for storing ASCII characters, including escape sequences.

\paragraph{bool}
A bool encapsulates true and false values.

\paragraph{string}
A string is a sequence of characters surrounded by double quotation marks. In AVL, string is mainly
used for outputting debug information to terminal.

\paragraph{index}
An index is used for identifying an element’s position in an array, and range of positions to for a
subarray. It is interchangeable with int. In addition to int, index can be displayed as “arrow”.

\subsubsection{Derived Types}

\paragraph{array}
An array is a data structure that let you store one or more elements consecutively in memory. Only
int is allowed to be the data type of elements. \verb"len" returns the length of an array. The index
of an array ranges from \verb"0" to \verb"len - 1". To declare an array, you need to either declare
its size or directly initialize this array.

There are three ways to initialize an array. One is the assign value to each elements, and the
default value is 0. Another one is to initialize an array with a comma-separated array surrounded by
\verb"{" and \verb"}". The other one is to assign another array or subarray to the newly declared
array. \verb"a[s:e]" defines a’s subarray starting at index s and ending at index e, left-closed,
right-open, which means \verb"a[s]" is included and \verb"a[e]" is excluded.

\subsubsection{Constants}

\paragraph{Integer Constants}
An integer constant is a sequence of digits, representing a decimal number.

\paragraph{Character Constants}
A character constant is a single character enclosed within single quotation marks.

\paragraph{Boolean Constants}
Boolean constants are either \verb"true" or \verb"false".

\subsection{Scopes}

\subsubsection{Lexical Scope}

The lexical scope of an identifier for a declared variable starts at the end of its declaration
immediately and lasts until its current scope exits. Duplicate variable names within the same scope
are not allowed.  

\subsubsection{Function Scope}

The scopes of variables declared inside a function persist through the function and end upon exiting
the function.

\subsubsection{Statement Block Scope}

The scopes of variables declared inside a statement block persist through the block and end upon
exiting the block.

\subsection{Syntax and Semantics}

\subsubsection{Expressions}

\paragraph{Primary Expressions}
Primary expressions are identifiers, constants, strings and parenthesized conditional expressions.

\begin{verbatim}
    primary_expression
        : IDENTIFIER
        | CONSTANT
        | TRUE
        | FALSE
        | CHAR_LITERAL
        | STRING_LITERAL
        | '(' conditional_expression ')'
        ;
\end{verbatim}

\paragraph{Postfix Expressions}
Postfix expressions includes function call, array elements, sub arrays, postfix operators. The
grammar of postfix expression is:

\begin{verbatim}
    postfix_expression
        : primary_expression
        | IDENTIFIER '[' conditional_expression ']'
        | IDENTIFIER '[' conditional_expression ':' conditional_expression ']'
        | IDENTIFIER '(' ')'
        | IDENTIFIER '(' argument_expression_list ')'
        | postfix_expression INC_OP
        | postfix_expression DEC_OP
        ;

    argument_expression_list
        : conditional_expression
        | argument_expression_list ',' conditional_expression
        ;
\end{verbatim}

Subarray is written in python style, for example \verb"a[2:5]" will return a new array
\verb"{a[2], a[3], a[4]}".  The postfix operators we support are \verb"++" and \verb"--". These two
postfix operators have the highest precedence.

\paragraph{Unary Expressions}
Unary Expression include unary operator and keyword len. len is a unary operator like sizeof in C to
get the length of an array. The different is len only return the number of elements in the array.
Let us see the grammar:
\begin{verbatim}
    unary_expression
        : postfix_expression
        | INC_OP unary_expression
        | DEC_OP unary_expression
        | unary_operator cast_expression
        | LEN '(' unary_expression ')'
        ;

    unary_operator
        : '+'
        | '-'
        | '!'
        ;
\end{verbatim}

In AVL, unary (prefix) operator includes ‘+’, ‘-’, ‘!’, ‘++’ and ‘--’. The have the second highest
precedence.

\paragraph{Types}
The type we support in AVL are:

\begin{verbatim}
    type_specifier
        : VOID
        | CHAR
        | INT
        | STRING
        | INDEX
        | BOOL
        | type_specifier '[' ']'
        ;
\end{verbatim}

\paragraph{Binary Operators}
The binary operators except for the assignment operator are listed according to precedence in the
following table. The closer to the top of the table an operator appears, the higher its precedence.
Operator with higher precedence are evaluated before operators with relatively lower precedence.
Operators on same line have equal precedence and are evaluated from left to right.

\begin{table}[htp]
  \centering
  \begin{tabular}{l|l}
    \hline
    Name & Operator \\
    \hline
    Multiplicative \hspace{2cm} & \verb"* / %" \\
    additive & \verb"+ -" \\
    relational & \verb"< > <= >=" \\
    equality & \verb"== !=" \\
    logical and & \verb"&&" \\
    logical or & \verb"||" \\
    \hline
  \end{tabular}
  \caption{Binary operators}
  \label{tab:bin_op}
\end{table}

\subparagraph{Multiplicative Operators}
\verb"*" and \verb"/" represent multiplication and division operators. Division with int operands
results in discarding the fractional portion of the result.

\verb"%" represents remainder operator. It divides one operand by another and returns the remainder
as its result.

\begin{verbatim}
    multiplicative_expression
        : unary_expression
        | multiplicative_expression '*' unary_expression
        | multiplicative_expression '/' unary_expression
        | multiplicative_expression '%' unary_expression
        ;
\end{verbatim}

\subparagraph{Additive Operators}
\verb"+" represents the summation of the left operand and the right operand.

\verb"-" represents the subtraction operator that returns the result of left operand subtract the right
operand.

\begin{verbatim}
    additive_expression
        : multiplicative_expression
        | additive_expression '+' multiplicative_expression
        | additive_expression '-' multiplicative_expression
        ;
\end{verbatim}

\subparagraph{Relational Operators}
The relational operators determine if one operand is less than (\verb"<"), greater than (\verb">"),
less than or equal to (\verb"<="), or greater than or equal to (\verb">=") another operand. It
returns a bool value true if the relation is true and false if the relation is false.

\begin{verbatim}
    relational_expression
        : additive_expression
        | relational_expression '<' additive_expression
        | relational_expression '>' additive_expression
        | relational_expression LE_OP additive_expression
        | relational_expression GE_OP additive_expression
        ;
\end{verbatim}

\subparagraph{Equality Operators}
The relational operators determine if one operand is equal to (\verb"=="), or not equal to
(\verb"!=") another operand. It returns a bool value true if the relation is true and false if the
relation is false.

\begin{verbatim}
    equality_expression
        : relational_expression
        | equality_expression EQ_OP relational_expression
        | equality_expression NE_OP relational_expression
        ;
\end{verbatim}

\subparagraph{Logical AND Operator}
\verb"&&" operator returns a bool value true if both operands evaluate to true or a bool equivalent.
Otherwise, it returns false.

\begin{verbatim}
    logical_and_expression
        : equality_expression
        | logical_and_expression AND_OP equality_expression
        ;
\end{verbatim}

\subparagraph{Logical OR Operator}
\verb"||" operator returns a bool value true if either of the two operands evaluate to true or a
bool equivalent. Otherwise, it returns false.

\begin{verbatim}
    logical_or_expression
        : logical_and_expression
        | logical_or_expression OR_OP logical_and_expression
        ;
\end{verbatim}

\paragraph{Conditional Operator}
Ternary conditional operator \verb"? :" can be thought of as shorthand for an if-else statement. If
the first operand evaluate to true, assign the value of the second operand to result. Otherwise,
assign the value of the third operand to result.

\begin{verbatim}
    conditional_expression
        : logical_or_expression
        | logical_or_expression '?' conditional_expression ':'
        conditional_expression
        ;
\end{verbatim}

\paragraph{Assignment Expression}
The left operand must be a postfix expression and the left operand must be a conditional expression
which has return value. It assign the value on its right to the operand on its left.

\begin{verbatim}
    assignment_expression
        : postfix_expression '=' conditional_expression
        ;
\end{verbatim}

\paragraph{Expression}
An expression is a construct made up of variables operators and function invocations. There are six
kind of expressions.

\begin{verbatim}
    expression
        : conditional_expression
        | assignment_expression
        | DISPLAY IDENTIFIER
        | HIDE IDENTIFIER
        | SWAP '(' IDENTIFIER ',' conditional_expression ',' conditional_expression ')'
        | PRINT print_list
        ;

    print_list
        : conditional_expression
        | print_list ‘,’ conditional_expression
        ; 
\end{verbatim}

Conditional expression and assignment expression are introduced above.

Display and hide expressions can decide whether to show a variable. If the variable is defined to be
displayed, the following execution about it inside the \verb"<begin_display>" and
\verb"<end_desplay>" block will be displayed. If it is hide, it will not be shown. The display
status of variables can be changed in the program.

Swap expression swaps two elements of one array. The first operand indicates the identifier of the
array. The next two operands indicate the index of the two elements.  The SWAP key word can help the
language to generate specific animation of swapping elements. Users can swap the elements by using
their own code, but they cannot get the animation by doing that.

Print expression prints information to the console for debugging. The \verb"print_list" consists of
a set of conditional expression separated by commas. The value of the conditional expression will be
joined and printed.

\subsubsection{Declarations}

Variables should be declared inside a function. A example of declaration can be:

\begin{verbatim}
    display int val = 5;
\end{verbatim}

Variable declarations are composed of three components:
\begin{itemize}
  \item Zero or more modifiers, such as display and hide
  \item The field’s type
  \item The field’s name and initializing value
\end{itemize}

\paragraph{Modifier}
The modifiers here indicates whether this variable can be displayed inside the
\verb"<begin_display>" \verb"<end_display>" block. It is an attribute of every variable. If you does
not specify the modifier when declaring, AVL will set hide as the default for it.

\paragraph{Types}
All variables must have a type. You can use primitive types such as int, char, index. Or you can use
reference types such as String.

\paragraph{Variable Names and Initializing Value}
Variables’ name should start with a letter and then followed by zero or more number, letter or
underscore. This is the same as in C. You can choose not to initialize a variable immediately in
declaration.

\paragraph{Return Value}
Although declaration in AVL is C styled. It does not return a value.  Here is the grammar of
declaration in AVL:

\begin{verbatim}
    declaration
        : type_specifier init_declarator_list
        | DISPLAY type_specifier init_declarator_list
        | HIDE type_specifier init_declarator_list
        ;

    init_declarator
        : declarator
        | declarator '=' initializer
        ;
\end{verbatim}

Similar to C, the declarator(left value) can be an identifier or an array. The initializer(right
value) can be an expression or a list of value inside a brace. An example is

\begin{verbatim}
    int array[] = { 1 , 2 , 3 }.
\end{verbatim}

Grammars for declarator and initializer are as follow:

\begin{verbatim}
    declarator
        : IDENTIFIER
        | IDENTIFIER '[' conditional_expression ']'
        | IDENTIFIER '[' ']'
        ;

    initializer
        : conditional_expression
        | '{' initializer_list '}'
        ;

    initializer_list
        : conditional_expression
        | initializer_list ',' conditional_expression
        ;
\end{verbatim}

Note that here we does not support 2-dimensional array in AVL.

\subsubsection{Statements}

A statement forms a complete unit of execution. There are five kind of statements which are
expression statement, declaration statement, display statement, control flow statement and compound
statement. The first four kind of statements can be nested in compound statement.

\begin{verbatim}
    statement
        : compound_statement
        | expression_statement
        | declaration_statement
        | display_statement
        | selection_statement
        | iteration_statement
        | jump_statement
        ;
\end{verbatim}

\paragraph{Expression Statement}
It is the most basic statement, which is an expression terminated by a semicolon or a single
semicolon.

\begin{verbatim}
    expression_statement
        : expression ';'
        ;
\end{verbatim}

\paragraph{Declaration Statement}
It is used to declare variables, which is formed by a declaration and a semicolon.
\begin{verbatim}
    declaration_statement
        : declaration ';'
        ;
\end{verbatim}

\paragraph{Display Statement}
It forms a block of statements between keywords \verb"<begin_display>" and \verb"<end_display>". The
execution of the statement list, which is a group of statements, in the block will be displayed to
the user.

\begin{verbatim}
    display_statement
        : BEGIN_DISPLAY statement_list END_DISPLAY
        ;

    statement_list
        : statement
        | statement_list statement
        ;
\end{verbatim}

\paragraph{Control Flow Statement}

\subparagraph{Selection Statement}

The if statement is the most basic control flow statements. It tells the program to execute the
certain section of statement only if the expression is true.

The if-else statement evaluates the expression and executes the first statement if it is true.
Otherwise it executes the second statement. The dangling else ambiguity is resolved by binding the
else with the nearest if.

\begin{verbatim}
    selection_statement
        : IF '(' conditional_expression ')' statement
        | IF '(' conditional_expression ')' statement ELSE statement
        ;
\end{verbatim}

\subparagraph{Iteration Statement}
The while statement evaluates expression and executes the statements if it is true. It continues
testing the expression and executing its statements until the expression evaluates to false.

The do-while statement evaluates its expression at the bottom of the loop instead of the top.
Therefore, the statements within the do block are always executed at least once.  In the for
statement, the first operand in the parentheses is executed once as the loop begins, which can be an
expression or a declaration. The second operand is termination expression, that when it evaluates to
false the loop terminates. The third operand is invoked after each iteration of the loop.

\begin{verbatim}
    iteration_statement
        : WHILE '(' conditional_expression ')' statement
        | DO statement WHILE '(' conditional_expression ')' ';'
        | FOR '(' expression ';' conditional_expression ';' ')' statement
        | FOR '(' expression ';' conditional_expression ';' expression ')' statement
        | FOR '(' declaration ';' conditional_expression ';' ')' statement
        | FOR '(' declaration ';' conditional_expression ';' expression ')' statement
        ;
\end{verbatim}

\subparagraph{Jump Statement}
The continue statement skips the current iteration to the end of the innermost loop’s body and
evaluates the expression that controls the loop.

The break statement terminates the innermost control flow statement.  The return statement exits
from the current function and returns to where the function was invoked. It can return a value or
nothing.

\begin{verbatim}
    jump_statement
        : CONTINUE ';'
        | BREAK ';'
        | RETURN ';'
        | RETURN conditional_expression ';'
        ;
\end{verbatim}

\paragraph{Compound Statement}

It is a group of zero or more statements between balanced braces and can be used for function
definitions and control flow statements.

\begin{verbatim}
    compound_statement
        : '{' '}'
        | '{' statement_list '}'
        ;
\end{verbatim}

\subsubsection{Function Definition}

A function consist of the return type, function name, parameter list surrounded by parentheses and
the body between braces. The parameter list can be empty. Parameters of array type are passed in by
reference, and others are passed in by value.

\begin{verbatim}
    function_definition
        : type_specifier  IDENTIFIER '(' parameter_list ')' compound_statement
        | type_specifier  IDENTIFIER '('  ')' compound_statement
        ;
\end{verbatim}

The parameter list consists of a set of type and declarator pairs which are separated by commas.

\begin{verbatim}
    parameter_list
        : parameter_declaration
        | parameter_list ',' parameter_declaration
        ;

parameter_declaration
        : type_specifier declarator
        ;
\end{verbatim}

\subsubsection{Program}

The translation unit consists of multiple function definition. So our language does not support
Macro and Global Variable in C. The whole function itself is a translation unit.

\begin{verbatim}
    translation_unit
        : function_definition
        | translation_unit fucntion_definition
        ;

    program
        : translation_unit
        ;
\end{verbatim}

\subsection{Complete Grammar}

\input{parser.y}
