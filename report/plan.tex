\section{Project plan}

\subsection{Management Process}

Our team was formed after the first class. During the early process of the project we met two to
three times a week to brainstorm ideas about the new language. We use three weeks to discuss what
language we want to implement. Finally, we combined several ideas of different team members and AVL
came out.

After we decide the language, we met once a week to discuss what we can do for the next step. We
separated the project into two part: the front-end and back-end, and they are developed parallelly.
The back-end is the library code which is used after AVL code was translated to C++ code. We began
early in implementing the back-end, since it need less knowledge about PLT when we started to write
code. The front-end process went on according to the what we learned from the course. The workloads
are divided to creating grammar, implementing scanner and parser, building abstract syntax tree,
implementing code generator and doing semantic check. 

We adopt GitHub to help us control the code version. We require each member push their code to the
master node after testing the code locally. At least the code can be compiled at that time.
Otherwise, the member need to create a branch on Github. It can make sure that the team member’s
work will not be slowed down by others. We assigned the work to each member according to the
modularity rule that each one was not working on a same file at one time. That can greatly enhance
the efficiency of the whole process. When someone find the bugs which is not managed by him, he can
open an issue on Github and everyone can receive the notify email immediately.

For each file, we did small tests directly by the one who was responsible to it. This step can
decrease the number of bugs as soon as possible. After we complete the whole project, we tested them
together by using the designed set of  test cases. We could found some special bugs, and discussed
together to fix it.

\subsection{Roles and Responsibilities}

\begin{table}[htp]
  \centering
  \begin{tabular}{|l|l|l|}
    \hline
    Team Member & Role & Responsibility \\
    \hline
    Qianxi Zhang & Project Manager & parser, AST, code generator \\
    \hline
    Shining Sun & Language Guru & grammar, semantic check, AvlTypes \\
    \hline
    Yu Zheng & System Architect & AST, AvlTypes, test cases \\
    \hline
    Qinfan Wu & System Integrator & symbol table, AvlTypes, AvlVisualizer, AvlUtils, runtime environment \\
    \hline
    Jiuyang Zhao & System Tester & scanner, parser, test cases \\
    \hline
  \end{tabular}
  \caption{Roles and responsibilities}
  \label{tab:roles}
\end{table}

\subsection{Implementation Style Sheet}

\begin{itemize}
\item
Keep code neat and clean. Use 4 spaces tab to indent.
\item
Make sure the code can be compiled when push the code to master node in GitHub.
\item
Push uncompilable code to new branches if needed.
\item
Push code regularly to let other team members keep track of the newest version.
\item
When one find a bug which cannot be fixed by himself, notify team members through GitHub.
\end{itemize}

\subsection{Timeline}

\begin{table}[htp]
  \centering
  \begin{tabular}{|l|l|}
    \hline
    Date & Milestone \\
    \hline
    01/22/2014 \hspace{1cm} & Team Formation \\
    02/16/2014 & Project AVL Confirmed \\
    02/24/2014 & GitHub Opened \\
    02/26/2014 & Whitepaper Completed \\
    03/09/2014 & Build Environment Initialized \\
    03/23/2014 & AvlUtils Implemented \\
    03/24/2014 & AvlVisualizor Implemented \\
    03/25/2014 & Scanner \verb"&" Parser Implemented \\
    03/26/2014 & Language Tutorial \verb"&" LRM Completed \\
    03/28/2014 & AvlInt and AvlArray in AvlType Implemented \\
    04/16/2014 & AvlChar in AvlType Implemented \\
    04/17/2014 & AvlBool in AvlType Implemented \\
    04/20/2014 & AST Implemented \\
    04/20/2014 & AvlString in AvlType Implemented \\
    04/26/2014 & Code Generator Implemented \\
    05/03/2014 & Test Cases Design Completed \\
    05/09/2014 & AvlIndex in AvlType Implemented \\
    05/10/2014 & Semantic Check Implemented \\
    05/11/2014 & Test Finished \\
    \hline
  \end{tabular}
  \caption{Timeline}
  \label{tab:timeline}
\end{table}

\subsection{Project Log}

Project log is appended in Appendix.
