\section{Language evolution}

In the original language proposal, we proposed a language that visualizes algorithms for teaching
and learning purpose. Actually it was me who initially came up with this idea, and that was how I
ended up with this role. Originally we had plenty of interesting and ambitious ideas, including
visualize Aho–Corasick string matching algorithm. But as the implementation process going, we found
that it was not easy to implement all of them because different algorithms requires different data
structures, and different data structures requires different methods for vitalization. For example,
it is very different to visualize an array than to visualize a finite automaton. Therefore, we
decided to start from the most basic structures, array. 

Without distracting thoughts, we focus on implementing array. We firstly had a case study on the
algorithms that use array as their primary data structures, like sorting algorithms, and tried to
extract common features of them. We then concluded that there need to have swap as a build-in
operation on arrays. For example, in compare-based sorting algorithms, it is essential to have swap
because the basic idea behind these algorithms is to compare two elements and adjust their
positions. Therefore, having swap as build-in operator makes it convenient for the programmer to
visualize array-based algorithms. 

Also, we thought that it is useful to give the programmer the right to decide which part of his
program to be visualized. This makes sense because although it is a cool idea to visualize
algorithms, it is possible that user may not want to visualize everything they write. For example,
as mentioned above, it is possible to visualize sorting algorithms. But in other algorithms, sorting
might be a sub-routine and should not be visualized. Therefore, we have defined
\verb"<begin_display>" and \verb"<end_display>" to enclose the part that needs to be visualized.
Also, for each variable, programmer can also choose to visualize it or not. 

Along the process of implementation, we also made some decisions on what not to be included in the
language. For example, we decided not to include pointers, because it is way to complicated to
implement. As another example, while a struct is very useful in C, we decided to not include that in
our language, because there is no way to correctly visualize every user-defined struct. 

\subsection{Compiler Tools}

We decided to use C++ as the target language for our compiler because that is the most familiar
language for all of the team member. 

Lex and Yacc are used as scanner and parser for our compiler, because it is compatible with C/C++,
and there are tons of tutorials, resource and documentations online for them. 

\subsubsection{Libraries}

The main functionality of our language is visualization. Therefore, we used OpenGL, freeglut, glew
glm and glfw as they are the most commonly used libraries in computer graphics. 

\subsection{Consistency}

Although there are some aggressive ideas in the white paper, in the LRM, the only structure left is
the array. Therefore, we did not had a chance to deviate from the LRM. Besides, we thought it is
nice to have our grammar as close to that of C as possible, because that will create a more
reasonable learning curve for programmers. Therefore, we largely referred to the C standard as we
writing our grammar. This also helps us to stick to our original idea, grammar wise. 
