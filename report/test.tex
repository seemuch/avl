\section{Test plan: Jiuyang Zhao}

\subsection{Tools}

AVL compiler uses GNU Autotools to generate makefiles automatically. Autotools provides the function
make check to run our tests cases automatically. By running this command, our makefile will run the
test cases sequentially one by another. It give the result into four classes. \verb"#PASS",
\verb"#FAIL", \verb"#XPASS" and \verb"#XFAIL". Each of them denotes the meaning that: the correct
test cases that passed(good case), the correct test cases but failed(bad case), the incorrect test
cases but passed(bad case) and the incorrect test cases that failed(good case).

After running  make check, we can get to statistics result of the successful rate of our test cases.
After the we run the that command, a log file will be generated for each test case to record the
output of stdout and stderr. So once that are unexpected result, we can check the log file and debug
easily.

\subsection{Relevant and Description}

Our test cases can be divided into seven classes, each classes corresponds to a part in AVL grammar.
They are:
\begin{itemize}
\item constant\_*: 
This is a most simple one. Just test if there are any error when we taking each data types as
constant.
\item declaration\_*:
In declaration part we test different way of declaration for data type. For example: adding keyword
‘display’, ‘hide’ or not, declare with initialization value or not, and different way in declare an
array.
\item expression\_*:
In expression part we test some operation in expression like swaping an array, or displaying/hiding
a variable in half way.
\item operator\_*:
In operator we test the basic operation for every data type, including postfix, unary and binary
operator. Because different data type are implement in different class at our back end, and we
overload every operator for them. So it is necessary to check these member functions.
\item display\_*:
For each object like AvlInt and AvlArray in our back-end, we use the function render to draw each
object and ready to print. We design part of test cases to check whether each kind of data type can
be displayed correctly.
\item function\_*: 
Here we test for parameter and return value for each data type. Also we test the function of
subarray and the displaying when functions are nested.
\item statement\_*:
Here we mainly aim to test for different of control flows like if, for and while. Also we have
tested for some basic scope.
\end{itemize}

\subsection{Selected Test Cases}

This is a simple test cases for declare an bool array:

\begin{verbatim}
    int main()
    {
        bool b6[5];
        display bool b7[5];
        hide bool b8[6];
        bool b9[3] = {true, true, false};

        return 0;
    }
\end{verbatim}

This is another test cases for for loop taken as looping variable.

\begin{verbatim}

int main(){
    /* empty, line 4 failed */
    for (int i = 0; i < 10; i++);

    for (int i = 0; i<10; i++) {

    }

    /* regular for */
    for (int i = 0; i < 10; i++) {
        print i;
    }

    /* changing i inside */
    for (int i = 0; i < 10; ) {
        print i;
        i = i + 2;
    }

    /* decleare i outside for */
    int i = 0;
    for (i; i < 10; i++)
        print i;
    for(i; i > 0;)
        print i--;

    /* testing continue, break */
    for (i; i < 10; i++) {
        if( i == 2)
            continue;
        else if (i == 5)
            break;
        print i;
    }

    /* testing nested for */
    for (i = 0; i < 3; i++) {
        for (int j = 0; j < 2; j++)
            print j;
    }

    /* testing return */
    for (i = 0; i < 10; i++){ 
        print i;
        if (i == 5)
            return 0;
    }

    return 0;
}
\end{verbatim}

\subsection{Notable Bugs Encountered}

\begin{enumerate}
\item We must add \verb"()" when we translate print into \verb"std::cout" because in c++ the
precedence of \verb"<<" is higher than some operator we need in AVL like \verb"||".
\item Because we translate int to our own defined type AvlInt, we must treat the \verb"int" in
\verb"int main().." differently to avoid the main function in c++ returning an \verb"AvlInt".
\item Our test cases in loop statement find out that we cannot handle empty loop before.
\end{enumerate}
