\section{Language Tutorial}

This section is a short tutorial of the Algorithm Visualization Language. In this section, we will
go through some example programs that demonstrate our language and show how to use our language to
visualize some simple algorithms.

This tutorial will focus on basic data types, expressions, control flows, inputs and outputs so that
users can get started with AVL quickly. In order to provide users a way to visualize algorithms by
writing C-style code, the syntax of AVL is very similar to C. Users who have a basic understanding
of C programming would be able to write AVL programs after reading this tutorial.

AVL is an intuitive educational programming language that concentrates on the algorithm teaching and
learning in the way of visualization. The output of AVL is a sequence of graphics or video, which
describes how an algorithm works. AVL provides varying degrees of flexibility and programmability.
It helps beginners understand a complicated algorithm in an intuitive way. It assists professionals
to visualize their self-design algorithms, where AVL provides visualization control interface.

AVL breaks the limitation of current similar languages. It is highly focused on the robust algorithm
guarantee as a visualization language. The shining breakthrough is to provide a series of
complicated low-level drawing operations, which is specially designed for algorithm visualization.

AVL is also an educational language. It provides simple and clear grammar with C++ style. Benefiting
from the easy-to-use features of AVL, the target users, teachers and students, are able to easily
demonstrate or learn the algorithms.

First, let us examine the input and output of the compiler.

\subsection{Input and Output}

The purpose of AVL is to help user create a program which visualizes a particular algorithm without
much effort. The user creates a \verb".avl" file as input, which contains AVL code that describes
the algorithm. Then the \verb".avl" file is compiled by our compiler and an executable is created.
When user runs the executable, they would see a window that displays the algorithm procedure.

We provide a program avl which acts as the compiler.

\begin{verbatim}
$ avl --help
Usage: avl [-h|-o|-t] file
Options:
    -h --help            Display this information
    -o --output=<file>   Compile and place the executable into <file>
    -t --translate       Translate the source files into c++

    Use at most one option at a time.
\end{verbatim}

The input filename should have the extension \verb".avl".

Assuming the input file is \verb"test.avl", the following command will create an executable file
with the default output name \verb"a.out":

\begin{verbatim}
    avl test.avl
\end{verbatim}

Then the user can run \verb"./a.out" to display the algorithm procedure.

The user is also allowed to specify the output name of the executable using the -o option:

\begin{verbatim}
    avl -o test test.avl
\end{verbatim}

For debugging purpose, if we want to examine whether the compiler works correctly, we could invoke
avl with -t option to translate the input file into a C++ file, which is the intermediate result:

\begin{verbatim}
    avl test.avl -t
\end{verbatim}

This would generate a C++ file with name test.cpp.

\subsection{Hello World}
A Hello World program in AVL creates images of the characters ``Hello world'' on the screen, and
images of an integer array on the screen. This program is for demonstration of the basic
visualization feature of the program.

\begin{verbatim}
    int main()
    {
        <begin_display>
            display char str[] = {'H', 'e', 'l', 'l', 'o', ' ', 'w', 'o', 'r', 'l', 'd'};
            display int a[] = {1, 2, 3, 4};
        <end_display>

        return 0;
    }
\end{verbatim}

As you may already notice, the code is similar to C/C++ code. This is actually a feature of AVL:
programmers do not need to spend too much time to learn new grammars. As long as a programmer knows
C/C++, she or he will have little obstacle picking up AVL.

Now let me briefly explain the program.  The first line declares a main function, which is the
entrance of any program. On line 2 and 5, there are a \verb"<begin_display>" and a
\verb"<end_display>". This means anything in between this two tags are to displayed on the screen.
On line 3 and 4, a char array and a integer array are declared, and \verb"display" keyword indicates
these arrays are displayable on the screen. Variables are not displayable by default, unless it has
a \verb"display" key word before it.  Notice that the \verb"display" keyword does not have to be put
before variables while being declared. A variable can be declared as non-display while later changed
to display.


\subsection{Types and Variables}

AVL is a strongly typed language. For the current version, it supports the following types:

\begin{verbatim}
    int
    char
    bool
    index
    string
    array
\end{verbatim}

Among them, array is the only derivative type. There could be array of ints, or chars, etc. Most of
them are the same with C/C++, while there are several things to be noticed.

\begin{enumerate}
\item
If a programmer wants to display a string on the screen, she should use array of chars, instead of
\verb"string". The type \verb"string" are supposed to be used to print things in terminal, e.g. for
debug use.

\item
The type \verb"index" is used for indexing elements inside an array. While it is similar to the
\verb"int" type , the purpose of having this type is to highlight certain elements inside an array.
Therefore, whenever we want to index an array, we should use \verb"index" instead of \verb"int".

\item
The type \verb"array" is not similar to C/C++ arrays in a number of ways.  First, the length of an
array in AVL can be obtained by calling \verb"len(arr)". Also, subarray of an array can be
conveniently obtained by calling \verb"arr[i : j]" , where \verb"i" and \verb"j" are two indices,
and the array obtained is the subarray of \verb"arr" that starts at index \verb"i" and ends at index
\verb"j-1". Lastly, if an array is declared without initialization, it is set to be the default
value of its type. For example, the following statements creates an array of five 0’s:

\begin{verbatim}
    int a[5];
\end{verbatim}

\end{enumerate}

\subsection{Display Hierarchy}

As described in the Hello World part above, a variable is set to be displayable only when it is
declared to be \verb"display". This can happen during or after the variable being declared. However,
the \verb"display" key word does not actually display things on the screen. What actually displays
things on the screen is the pair of \verb"<begin_display>" and \verb"<end_display>". Things between
this two tags are displayed (if they are declared to be \verb"display").

A intuitive question to ask is, what if the tag pairs are nested?  This behaves differently in two
cases. The first case is, the tag pairs are nested within one function. For example:

\begin{verbatim}
    int main()
    {
        <begin_display>
        <begin_display>
            display char str[] = “Hello world!”;
            display int a[] = {1, 2, 3, 4};
        <end_display>
        <end_display>

        return 0;
    }
\end{verbatim}

In this program, the tag pairs are nested. In this case, the inner pair are ignored.

The second case is, the nesting happens in another function. For example:

\begin{verbatim}
    void quicksort(int a[])
    {
        if (len(a) <= 1)
            return;

        display index i = -1;
        display index j = 0;
        display index k = len(a) - 1;

        <begin_display>
            while (j < k) {
                if (a[j] >= a[k]) {
                    j = j + 1;
                }
                else {
                    swap(a, i + 1, j);
                    i = i + 1;
                    j = j + 1;
                }
            }
            swap(a, i + 1, k);

            quicksort(a[0 : i+1]);
            quicksort(a[i + 2 : k + 1]);
        <end_display>
    }
   
    int main()
    {
        <begin_display>
            display int a[] = {5, 2, 3, 6, 1, 7, 4, 9, 8};
            quicksort(a);
        <end_display>

        return 0;
    }
\end{verbatim}

In this program, there is a display tag pair inside \verb"main" function. In between this tag pair,
another function \verb"quicksort" is called, within which there is another display tag pair. In this
case, only the things inside the \verb"quicksort"’s display tag pair are displayed, while things
inside \verb"quicksort" but outside the display tag pair are not displayed.  That means, if there
were not a display tag pair inside \verb"quicksort", then nothing inside \verb"quicksort" is
displayed.

\subsection{Sample Code}

One more piece of sample code is listed below:

\begin{verbatim}
    int main()
    {
        display int a[] = {5, 2, 7, 3, 6, 8};

        <begin_display>
            for (display index i = 1; i < len(a); i = i + 1)
            {
                display index j = i - 1;

                while (j >= 0 && a[j] > a[j + 1])
                {
                    swap(a, j + 1, j);
                    j = j - 1;
                }
            }
        <end_display>

        return 0;
    }
\end{verbatim}

This code above shows a bubblesort for an integer array.
