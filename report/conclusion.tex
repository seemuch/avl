\section{Conclusions}

\subsection{Lessons Learned as a Team}

\begin{enumerate}
\item
One of the most important things is to invent a great language idea. The idea should be
interesting enough and can be done in given a semester. Also, the team should be able to estimate
the total amount of work before making the plan for such a semester-long project.

\item
Meetings regularly to keep every team member up to date with bi-weekly tasks. Keep coding every
week improves efficiency and productivity. Each team member should feel a greater sense of
responsibility for his/her scope. In addition, each team member should finish his/her task in time.

\item
Using version control systems appropriately greatly helps development of a large project.
\end{enumerate}

\subsection{Lessons Learned by Each Team Member}

\subsubsection{Jiuyang Zhao}

From this project. I have done the task I never covered before. Testing is not as easy as most
people imagine. First, To write a good test code we need to deeply know the target to the program we
write. So I take part in designing grammar, and backend function coding to know both front-end and
back-end things in a compiler, but I found this knowledge is still far from enough and often I still
have trouble to locate the bugs.  Second, the concept of unitest is difficult to implement. In this
project we use autotool in GNU to generate automatic test-bench. However it is not atomic enough to
cope with every corner of our program. Most of test cases I write was to test the every production
in grammar, and the different function in our back-end class. However, to test the function like
scope and memory leak is still very difficult. 

Also designing a good language is also very tough because there are no existing standard and method
to designing context free grammar. We have some technique to generate regular expression. But we
don’t have too much way to deal with CFG. The only technique we have is some basic things like
remove left reduction, factoring and disambiguous a grammar. With the powerful modern tools like
bison, this problem will be solved automatically, and how to design a grammar still remains a core
problem. The lesson we learn from this is trying to use the CFG from existent language such as C and
python. There CFG are already well defined so we don’t need to do most of task our self to write a
concise programming language.

\subsubsection{Qianxi Zhang}

First of all, the project helped me to have a deeper understanding of the knowledge we learned in
the course. As we discussed how to implement each part of the project a little bit earlier than the
course, we may think the question by ourselves. After that, we can compare our ideas with the
algorithm taught in class which helped a lot for grasp the outstanding points of the algorithm.
The most important things I learned from the project is about team working.

\begin{enumerate}
\item Creative Power: I really appreciate the creative power of the team, which I would never have by
working by myself. The first several ideas about the project were trivial or had already been
implemented by other teams in the past years. For example, language to solve the mathematical
problems or graph theory problems, language to generate movies or language to teach students.
Although each idea seemed simple, but when we combined them together, a novel language AVL came out.
I think it is worth to spend a long time to do brainstorming and don't give up when you feel
depressed. Discussing and exchanging ideas will get magic result and a great idea is half the battle
for a team.
\item Listening and Learning from Others: When implementing our project, we may often have
	different ideas with each other. Instead of consisting on our own ideas, the better way is to
	listen to others, and try to understand their core intention of the ideas. Then we can discuss
	the advantages and disadvantages of the ideas and finally find what is truly best for our
	project. Moreover, each member have their own talent, and we can learn a lot from others. For
	example, the rigorous coding style, the knowledge to set up the complex project environment or
	even basic skills such as how to use VIM efficiently.
\item Coding: For coding part, we learned a lot about version management. We emphasized the
	importance of uploading code after tested locally or at least can be compiled. Otherwise, others
	may be unable to debug their new code. Another way to avoid this problem is to let different
	member manage different file at same time and the interface of each module should be well
	defined earlier. Moreover, discussing frequently can help us catch up with others steps. Since
	we manage different parts of the project, if we have different understanding of the newest
	feature of the language, it may lead to bugs between the modules.
\end{enumerate}

\subsubsection{Qinfan Wu}

It is relatively hard to building a project from scratch. Also, we need to learn how to work in
a team of five, how to communicate with each other and build a big project in parallel. A few
things that worth noting are as follows.

\begin{enumerate}
\item
Have a good design and understanding of the project before coding. Keep modular design and
always make sure that you can add more features and make modifications without unnecessary
effort.
\item
Every time you implement a small feature, you need to do enough testing by creating some test
cases, then commit to the git repository. Also, it is hard to create test cases that cover all
the cases.
\item
Split up the whole project into appropriate modules. First define the interface between all
the modules and then implement each functionality.
\item
Working in a team with several people is more productive than working alone. Also, having
roles assigned helps divide the task into appropriate parts.
\item
Automate the building and testing process as much as you can. Almost all of them can be done
by Makefiles and bash scripts.
\item
All the code must be clearly organized and well commented.
\end{enumerate}

\subsubsection{Shining Sun}

This is one of the largest project that I have worked on. I had a enjoyable time, also a hard
time. Also, this is the largest team I have ever been in. Although suffering from time to time,
I did learn  a lot from this project. 

\begin{enumerate}
\item
Plan well. One of the lessons that I learned from this project is that always plan ahead of
time, especially large projects like this. Planning is good for all aspects. Time wise, a good
plan can save us from rushing towards the deadline. Work load wise, a good plan can keep
everybody in the team involved, and have a reasonable work load such that nobody can sit back
and take everything for granted, and nobody needs to have to much burden on his shoulder. 
\item
Execution. A good play needs good execution. Even with a perfect plan, if nobody follows it,
it still does not make any sense. There are tons of ways to keep the execution of the team. For
example, there could be some reward for people finish their work on time, and punishment for
those who do not. 
\item
Communication. I did not realize the importance of communication until I participated in this
project. It is so important and not so easy to understand each other’s idea. We had a lot of
arguments just because people did not understand each other. I think everybody agrees that
meeting regularly is very important. But I also learned that meeting with an agenda is also very
important. At the beginning of each meeting, an agenda should be firstly drafted. This is
essential for a meeting to be concise and time-saving.
\item
Coding. Coding is of course the main part of the project. With such a large project, I
learned that coding in good structure and style is very important. We have written about 7000
lines of code, and if the structure of the code is bad, it is almost impossible to maintain the
code. This is also very important for my future career, because in real-world industry, 7000
lines of code is actually not that much. Software nowadays can have millions lines of code. If
there is not a good structure and documentation, there is noway to maintain such a massive
project. 
\end{enumerate}

\subsubsection{Yu Zheng}

Building a project from scratch was quite an interesting experience. Also designing and
implementing a translator was pretty helpful for understanding programming languages. Through
this project, I learned how a source program was translated into an executable  file. More
importantly, I learned how a team worked together, and how a team should work efficiently and
effectively.

\begin{enumerate}
\item
    	Start early. Weekly meeting is essential, no matter how unnecessary you think it would
		be. Get prepared before meeting, and a well organized meeting is quite efficient.
\item
	    Working with talented teammates widened my horizon. They are not only classmates, but
		also teachers that encouraged me a lot. Working together can be very creative and
		productive. 
\item
		Project control tools, e.g. Automake, are life-saver. Generating projects
		automatically and programmatically guarantees the development process. Of course, a
		version control tool e.g. GibHub, is crucial, especially when the team size is larger
		than 3.
\item
		Nothing would be sweeter and cooler than solving an issue. 
\item
		You will never regret to take this course. This project is full of fun and really
		educative.
\item
		Teamwork is the most interesting and crutial part of this project. Believe your teammates and you will never
		walk alone.
\end{enumerate}

\subsection{Advice for Future Teams}

\begin{enumerate}
\item
Invent a good language idea.
\item
Use an appropriate version control system. Git/GitHub is a good choice.
\item
Use Autotools to generate and maintain Makefiles. Automake has good support for
Lex/Yacc. Autotools are also helpful for running test cases automatically.
\item
Given the time, don’t create too complicated grammars, which makes semantic check
much harder. Concentrate on the core functionality.
\item
Follow good coding style. Set ‘-Wall -Wextra -Werror’ for gcc and treat all warnings
as errors.
\item
Good communication between all team members is important.
\end{enumerate}

\subsection{Suggestions for the instructor on what topics to keep, drop, or add in future
courses}

\begin{enumerate}
\item
We want to learn more detailed information about semantic check, especially type
checking, scope checking, and symbol table.
\item
More detailed information about SDT is desired.
\item
We are curious about back end of the compiler. More back end topics are desired.
\item
We want to know more technique on how to design a context free grammar we want.
\end{enumerate}
